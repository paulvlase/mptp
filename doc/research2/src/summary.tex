
The \emph{swift} protocol is a multiparty content-centric protocol that aims to disseminate content among a swarm of
peers. This paper proposes an approach for the optimization of the currently \emph{swift} protocol. The integration of
the communication in the kernel space as a multiparty transport protocol that is solely responsible for getting the
bits moving improves the over all protocol performance. It ensures maximum efficiency of data transfer by decreasing
switches between user and kernel space and eliminating some performance penalties due to context switches.

\subsection*{Directions for Future Work}


After we complete the implementation and the functional tests, we want to test extensively our new features in a real
environment. We plan to do stress tests using a cluster. This tests will help us to make an overview about our
implementation and we could compare with the user-space implementation of the \emph{swift} to determine exactly what
performance we encountered. If the results are satisfactory, we will continue to optimize our program and we will add
new features.

It will be very useful to have a real application on top of the \emph{swift} protocol. If not, one solution would be to
port an application strictly for this task. This would give us the opportunity to extend and refine our implementation,
and also to extend the library API.

