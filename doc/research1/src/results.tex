%\section{\fontfamily{phv}\selectfont{\large{\bfseries{PRELIMINARY RESULTS}}}}

Our main focus when modifying the \emph{swift} implementation is to have an impact on time performance. With a
communication
protocol the greatest latency is usually generated by waiting for the results from the network. The multiparty
communication model already takes care of this, so the next best thing is to enhance the application time. We are doing
this by decreasing the time penalties due to context switches between user space and kernel space. The main idea is to
reduce the number of system calls made from user space into the kernel. This implicitly reduces the number of
preemption moments.

Another benefit is achieved by providing a user friendly API, through our library, that would simplify the
communication between user and kernel space. The common implementation of sockets provides support for peer to peer
communication, while \emph{swift} needs a multiparty communication, from one peer to many peers. This type of socket
implementation will be achieved with our \emph{swift} sockets. The library will ensure transparency and an abstraction
level of this communication.