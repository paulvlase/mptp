%\section{\fontfamily{phv}\selectfont{\large{\bfseries{SWIFT PROTOCOL DESCRIPTION}}}}

Most features of the \emph{swift} protocol are defined by its function as a content-centric multiparty transport 
protocol. A significant difference between \emph{swift} and the TCP protocol is that TCP possesses no information
regarding what data it is dealing with, as the data is passed from the user-space, while the \emph{swift} protocol has
data fixed in advance and many peers participate in distributing the same data. Because of this and the fact that for
\emph{swift} the order of delivery is of little importance and unreliability is naturally compensated for by redundancy,
it entirely drops TCP's abstraction of sequential reliable data stream delivery. For example, out-of-order data could
still be saved and the same piece of data might always be received from another peer.

Being implemented over UDP, the protocol does its best to make every datagram self-contained so each datagram could be 
processed separately and a loss of one datagram must not disrupt the flow. Thus, a datagram carries zero or more
messages, and neither messages nor message interdependencies should span over multiple datagrams. 

The verification of data pieces is realize using Merkle hash trees\cite{merkle}, \cite{merkle-ext}. That means that all
hashes necessary for verifying data integrity needs to be put into the same datagram as the data. For both use cases,
streaming and downloading, an unified  integrity checking scheme that works down to the level of a single datagram is
developed. As a general rule, the sender should append to the data some meta-data represented by the necessary hashes
for the data verification. While some optimistic optimizations are definitely possible, the receiver should drop data if
it is impossible to verify it. Before sending a packet of data to the receiver, the sender inspects the receiver's
previous acknowledgments to derive which hashes the receiver already has for sure. 

The data is acknowledged in terms of binary intervals, with the base interval of 1KB "packet". As a result, every 
single packet is acknowledged logarithmic number of times. This mechanism provides some necessary redundancy of the
acknowledgements and sufficiently compensates the unreliability of the datagrams. 

The only function of TCP that is also critical for \emph{swift} is the congestion control. To facilitate delay-based 
congestion control an acknowledgment contains besides the dimension of the file received from its addressee a timestamp.

Binary intervals numbering is done in the order of interval's "center", ascending, namely:


%\hspace*{3.75cm}               7
%
%\vspace*{-0.3cm}
%\hspace*{2.7cm}  3 \hspace*{1.65cm}  11
%
%\vspace*{-0.22cm}
%\hspace*{2.2cm} 1 \hspace*{0.6cm} 5 \hspace*{0.7cm} 9 \hspace*{0.8cm} 13
%
%\vspace*{-0.17cm}
%\hspace*{1.85cm} 0 \hspace*{0.2cm} 2 \hspace*{0.2cm} 4 \hspace*{0.1cm} 6 \hspace*{0.2cm} 8 \hspace*{0.1cm} 10
%\hspace*{0.1cm} 12 \hspace*{0.1cm} 14

\image[scale=0.28]{img/tree}{img:tree}{Binary interval tree}

Suppose, the receiver had acknowledged the first binary interval, then it must already have uncle hashes 5, 11 and so on. 
That is because those hashes are necessary to check the packets of the first two kilobytes acknowledged against the 
root hash. Then, hashes 3, 7 and so on must be also known as they are calculated in the process of checking the uncle
hash chain. Hence, to send the 12 binary interval, which represents the 7th kilobyte of data, the sender needs to
prepend hashes for binary intervals 14 and 9. This are the only hashes needed to check the against hash 11 which is
already known to the receiver.

The sender may optimistically skip hashes which were sent out in previous (still unacknowledged) datagrams. It is an 
optimization trade off between redundant hash transmission and possibility of collateral data loss in the case some
necessary hashes were lost in the network so some delivered data cannot be verified and thus has to be dropped. In
either case, the receiver builds the Merkle tree on-demand, incrementally, starting from the root hash, and uses it for
data validation.

The concept of peak hashes enables two cornerstone features of \emph{swift}: download and streaming unification and
file size proving. Formally, peak hashes are hashes defined over filled binary intervals, whose parent hashes are
defined over incomplete, not filled, binary intervals. Filled binary intervals is a binary interval which does not
extend past the end of the file, or, more precisely, contains no empty packets. Practically, we use peaks to cover the
data range with logarithmic number of hashes, so each hash is defined over a "round" aligned $2^k$ interval.

The classical problem of keeping huge bitmaps predominantly consisting of long ranges of zeros and ones is most often 
encountered in file systems (free space tracking) and network protocols (transmission progress tracking). For this
problem three common solutions are available: plain bitmaps, extent lists and extent binary trees. Bitmaps are simple
but have high fixed space requirements. Lists are able to aggregate solid ranges, but they don’t scale well with regard
to search. Extent binary trees are able of aggregation, allow scalable search, but have high overhead and extremely bad 
worst case behavior, potentially exploding to sizes a couple orders of magnitude higher than plain bitmaps. The latter
problem is sometimes resolved by ad-hoc means, e.g. by converting parts of an extent tree to bitmaps. Another possible
workaround is to impose a divide-and-conquer multilayered unit system (BitTorrent \cite{bittorrent}).

\emph{Swift} solution is a new data structure named “binmap”\cite{binmaps}, a hybrid of bitmap and binary tree, which
resolves the shortcomings of the extent binary tree approach. Namely it has lower average-case overhead and as it is
tolerant to patchy bitmaps, its worst-case behavior is dramatically better.

In the next table we will discuss a comparison between \emph{swift} application and BitTorrent. We take into
consideration the most important features that we have in \emph{swift} and BitTorrent.   

\begin{center}
  \begin{tabular}{|c|c|c|}
	  \toprule
	  \centering{Properties} & \multicolumn{2}{|c|}{Protocols} \\
	  \cmidrule(r){2-3}
	  & Swift & BitTorrent \\
	  \midrule
	  Listening Ports  & 1 & 1 \\
	  \midrule
	  Sockets create for getting M pieces & 1 & M \\
	  \midrule
	  Packets receive for sending 1 piece & 1 & 1 \\
	  \midrule
	  Packets send for sending 1 piece & 1 & $Nr_{fails} + 1$\\
	  \midrule
	  Packets send for getting 1 piece & N & $Nr_{fails} + 1$ \\
	  \midrule
	  Packets receive for getting 1 piece & N & $Nr_{fails} + 1$ \\
	  \midrule
	  Util packages when getting 1 piece & 1 & 1 \\
	  \midrule
	  Recall posibility  & $\approx$$\frac{100}{2^{N}}\%$ & $0\%$ \\
	  \midrule
	  Time to getting M piece & $M * T_{fastest}$ & $T_{slowest}$ \\
	  \midrule
	  Bandwidth used   & high usage & high usage \\
	  \bottomrule
  \end{tabular}
\end{center}

This comparison table highlights the advantages of \emph{swift} over BitTorrent. \emph{Swift} uses a socket per file,
whereas BitTorrent opens as many as M sockets. This is particularly useful for big files. Also, if BitTorrent connects
to a lower seeder the waiting time will be bigger compared to \emph{swift}. 

In the case of transferring many small files, BitTorrent will be better than \emph{swift}. Another case when
\emph{swift} is less than BitTorrent is for the first transfer between a peer and a seeder. Both of them use the same
bandwidth, but the overall performance appears to be better for \emph{swift} in the Internet. 



The second table highlights the main differences between three transport protocols UDP, TCP and \emph{swift}. In
a perfect network UDP is the fastest way to transfer data, but in a real network problem arises because of corrupt
packets or failed transfers. TCP ensures the correct transmission and because of this it's slower than UDP. \emph{Swift}
tries to make a compromise between UDP and TCP. \emph{Swift} is a connection-less protocol, but checks the validity of
the received data. Also, transfers are made between more seeders, not just between a server and a host like TCP does.
Thus, in a real network, \emph{Swift} is more secure than UDP and faster than TCP.


\begin{center}
    \begin{tabular}{|c|c|c|c|}
        \toprule
        Properties & \multicolumn{3}{|c|}{Protocols} \\
        \cmidrule(r){2-4}
         & Swift & UDP & TCP \\
        \midrule
        Sockets  & 1 & 1 & 1 \\
        \midrule
        Listening Ports  & 1 & 1 & 1 \\
        \midrule
        Packets received for sending data  & 1 & 1 & 1 \\
        \midrule
        Packets sent for sending data  & 1 & 1 & $Nr_{fails} + 1$ \\
        \midrule
        Packets sent for getting data  & N & 1 & $Nr_{fails} + 1$ \\
        \midrule
        Packets received for getting data  & N & 1 & $Nr_{fails} + 1$ \\        
        \midrule
        Util packets when getting data  & 1 & 1 & 1 \\
        \midrule
        Recall posibility when getting data & $\approx\frac{100}{2^{N}}\%$ & $75\%$ & $0\%$ \\
        \midrule
        Usability when getting data & $\frac{1}{N}\%$ & $100\%$ & $100\%$ \\
        \midrule
        Bandwidth used   & high usage & low usage & medium usage \\
        \bottomrule
    \end{tabular}
\end{center}    

